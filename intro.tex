\section{Introduction}
Along with the introduction of cryptocurrency in the form of Bitcoin~\cite{bitcoin}, the wallet was introduced as the primary way to interact with the cryptocurrency.

Usability is one of the most important obstacles in the adoption of cryptocurrencies~\cite{meiklejohn2018top}.

\paragraph{Our contributions.} In this work, we:
\begin{itemize}
    \item Define the purpose of a cryptocurrency wallet: to create and broadcast valid transactions, to know the balance and to know the transaction history.
    \item Provide detailed descriptions of how wallets in practice work for both transparent and private blockchains.
    \item Compare the known wallet protocols on their performance and security characteristics.
    \item Define for the first time the functionality of light clients, which are commonly conflated with light wallets.
    \item Provide a construction for light wallets based on a light client.
\end{itemize}

\section{Definitions}
\begin{definition}
    A cryptocurrency wallet $W_\text{seed}$ is a pair of algorithms (\textsf{Sync}, \textsf{History}, \textsf{Spend}).
    
    $\mathsf{Sync}(\text{state}) \rightarrow \text{state}'$: Takes a user's cryptocurrency seed and the wallet state$'$ and produces a new wallet state' after interacting with network participants. We denote the initial state as $\varepsilon$.
    
    $\mathsf{History}(\text{state}) \rightarrow 2^\textsc{TXs}$: Returns the transaction history for the wallet seed.
    
    $\mathsf{Spend}(\text{state}, \text{txDescription}) \rightarrow \text{tx}$: Takes a transaction description txDescription and returns a valid transaction tx that spends from funds owned by the wallet's seed.
\end{definition}

\paragraph{Computational Complexity.} Computation to restore from a seed.
\paragraph{Communication Complexity.} Bandwidth to restore from a seed.
\paragraph{Privacy.} Do we disclose any private key to an external participant or server?

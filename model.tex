\section{Our Model}
A cryptocurrency wallet facilitates the transfer of funds between individuals. The wallet contains the private keys that can be used to spend the user's cryptocurrency. With those private keys, it should be able to fulfill at least the following roles.

\begin{enumerate}
    \item Present the user with their balance.
    \item Allow the user create new valid transactions, given a description.
    \item (Optionally) Present the user with their transaction history.
\end{enumerate}

The lifecycle of the wallet is as follows. First the user generates a new seed or recovers from an existing seed, which is their secret. The wallet then performs an initial synchronisation by using the network, in order to obtain everything necessary for it to be able to perform the aforementioned functions. We denote this synchronisation step as $\text{state} \gets \text{Sync}_{\Gen, \mathcal{S}}(pk)$, where $\Gen$ is the genesis block of the blockchain and $\mathcal{S}$ a set of servers the wallet is allowed to interact with. After this initial synchronisation is complete, the wallet keeps using the network to stay up to date with relevant relevant events such as the user receiving a new transaction. The user may shut down the wallet and start it at some other point in time, when the wallet will attempt to catch-up to the network in order to be able to provide once again the aforementioned functionality.

The user may wish to create a new transaction. They specify a \emph{transaction description} to the wallet which includes the recipients and the amounts of cryptocurrency the users wishes to send to each one. To obtain the final valid transaction invokes a function $tx \gets \text{Fund}(sk, \text{state}, \text{description})$ which makes use of the user's secret key $sk$, the wallet state obtained from the Sync function and the provided transaction description.

We model the transaction history as a function extracting it from the wallet state, namely $tx_1, \dots, tx_y \gets \text{History}(pk, \text{state})$.

\begin{definition}[Wallet Protocol]
A \emph{wallet protocol} $\pi_\text{wallet}$ with respect to some server protocol $\pi_\text{wallet-server}$ that each server $s \in \mathcal{S}$ implements and a blockchain rooted in $\Gen$ is defined as a tuple of functions $\tpl{\text{Sync}_{\Gen, \mathcal{S}}, \text{Fund}, \text{History}}$.
\end{definition}

\paragraph{Seeds.}
Bitcoin wallets traditionally generate a different address for every transaction description to use for the tranasction's change output. This however meant that the user needed to periodically backup all their newly generated secret keys or risk part or all of their coins. A solution to this was presented in 2012 in the form of ``hierarchical deterministic wallets''~\cite{bip32}. A single secret called the \emph{seed} is used to deterministically generate all future secret keys of the wallet. It is thus necessary only to backup this seed for a user to be able to retain all their funds, no matter how long they make use of the wallet. Whereas wallets in practice usually restore from a seed, we adopt the simplification that the wallet is called to restore from a signature scheme secret key. Restoring from the seed can be thought of as invoking the Sync protocol many times from the first derivable address of the seed until we reach an address that seems unused.
